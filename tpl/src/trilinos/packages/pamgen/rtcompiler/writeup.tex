\documentclass{article}
\usepackage{amsmath}
\usepackage{graphicx}
\advance\textwidth by1.4in\advance\oddsidemargin by-0.7in
\advance\textheight by1.1in\headheight 0pt\topskip 0pt\headsep 0pt\topmargin 0pt
\begin{document}
\thispagestyle{empty}
\begin{center}
  \large\bf Runtime Compiler(RTC) \\
  \large\bf By Rich Drake and James Foucar \\
  \bigskip
\end{center}


\section*{Tables of Contents}
\begin{itemize}
  \item 1. Introduction
    \begin{itemize}
      \item 1.1 What is RTC?
      \item 1.2 Motivation for RTC
      \item 1.3 Abilities of RTC
      \item 1.4 Implementation of RTC
    \end{itemize}
  \item 2. Usage
    \begin{itemize}
      \item 2.1 Using RTC as a user-defined initial condition
      \item 2.2 Using RTC as a library
      \item 2.3 The RTC language
        \begin{itemize}
          \item 2.3.1 Operators
          \item 2.3.2 Control Flow
          \item 2.3.3 Line Structure
          \item 2.3.4 Variables
          \item 2.3.5 Math
          \item 2.3.6 Strings
          \item 2.3.7 Printf
          \item 2.3.8 Comments
          \item 2.3.9 Unsupported Features
          \item 2.3.10 Examples
        \end{itemize}
      \item 2.4 Using RTC and APREPRO
      \item 2.5 Adding Runtime-bound functions
      \item 2.6 Recent Changes
      \item 2.7 Troubleshooting
    \end{itemize}
\end{itemize}

\section{Introduction}

\subsection{What is RTC?}

RTC is an independent, self-contained library that compiles a string into
a set of in-memory data structures that can be executed. This gives users the
ability to runtime-compile a small program, pass in arguments to the program,
and execute this code all during runtime. If the client of RTC is getting this
string from a file, the file can be changed, potentially causing different
behavior in the client without the client having to be recompiled.

\subsection{Motivation for RTC}

RTC was created in order to provide users a way to express methods in the
input decks for physics codes. One common way which this ability is used
is for users to define the initial conditions for variables (density,
temperature, etc) as a function of the x,y, and/or z coordinates of the
entity (node, element, etc) whose property is being defined. This gives users
a more powerful way to define their initial conditions without requiring a
significant increase in the complexity of the input file parser. \\

\noindent
RTC is also useful for describing boundary conditions, source terms,
material properties, and any other independent variable.

\subsection{Abilities of RTC}

RTC allows users to set up a function with any number of arguments. Argments
may be of the type int, float, double, or char. Arguments may be arrays or
scalars. Arguments can be passed in by reference or by value (one caveat is
that arrays must be passed by reference). Once RTC is notified of the names
and types of a function's arguments, the body of a function (C-code packed into
a string) can be passed in and it will be compiled. The compiler will detect
any errors and provide the exact line at which the error occured. Errors are
returned to the user as a string. If the empty string is returned, then there
were no errors. At any point in time, the user can provide values for the
arguments they intend to pass into their RTC function. Once all the argument
values have been provided, the function can be executed. The function can
be indefinately re-executed with different argument values.

\subsection{Implementation of RTC}

\begin{center}
\includegraphics[scale=0.36,angle=270]{diag1}
\includegraphics[scale=0.36,angle=270]{diag2}
\end{center}

\noindent
The first diagram shows how the code is organized. Each individual line is
compiled into postfix and stored into the Line class. The Block classes
contain a collection of lines. The second diagram shows how individual
entities found in a line are organized.

\noindent
The RTC library has many classes, but only a few of them are interesting.\\

\noindent
The Function class provides the user interface to the library. It also
maintains the argument collection and ensure that all arguments are specified
before compiled and that their values are filled before execution. When a
function body is given to Function for compilation, it gets sent to
Tokenizer for tokenization and the tokens are then given to NormalBlock so
it can create its subcomponents (lines and sub-blocks). Execution is simple;
the top NormalBlock is told to execute. See below for more on blocks.\\

\noindent
The Tokenizer class contains the first level of the parser. It transforms
a raw string into collections of Tokens. Tokenizer parser a string by looking
at runs of characters. For example, if the first character it sees is numeric,
the parser knows it has encountered a number and will enter a loop, grabbing
numeric characters until there aren't anymore and then creating a token with
the grabbed characters. The grammar of the C-language is loosely enforces by
a comes-after operation. After the type of a token is verified, we check that
this type of token may follow the type of token that preceded it. While this
system may not be quite as robust as using grammar trees, it is much easier
to implement and catches the vast majority of grammar mistakes. The Tokenizer
class also provides convenient ways of iterating through its tokens. \\

\noindent
The Block class is a parent to the many types of sub-blocks (WhileBlock,
ConditionalBlock, ForBlock, etc). A block represents all the executable
items (lines and subblocks) within a \{ and \}. A block knows all the
varaibles that are available within its scope. During compilation, a block
creates its executable items by looking at the tokens it is given. For
example, say the following string was given to library to compile:
{\ttfamily \begin{verbatim}
int i = 0;                      \\1
int j = 0;                      \\2
if (i == j) {                   \\3
  int a = 2;                    \\4
  print(j);                     \\5
  for( i = 0; i < 10; i=i+1) {  \\6
    j = j + 2;                  \\7
  }                             \\8
}                               \\9
print(j);                       \\10
\end{verbatim} }
\noindent
The top block would be a NormalBlock that would contain the entire function.
Its list of executable items would be \{line 1, line 2, a ConditionalBlock
beginning at line 3, and line 10\}. The ConditionalBlock would contain its
condition statement $(i == j)$ and the executable items \{line 4, line 5,
and a ForBlock beginning at line 6\}. The ForBlock will contain its three
special statements: the initialization statement $i = 0$, the conditional
statement $i < 10$, and its post-processing statement $i=i+1$ and, of course,
its executable items: \{line 7\}. \\

\noindent
During compilation, a block will iterate through the tokens. If it sees a
block-opening statement (if, while, etc), it will create the appropriate
block and let the new block takeover. Once the new block returns, the original
block can assume that the new block iterated until it encountered a \} that
matched with its opening \{. This means that all executable items that follow
once again will belong to the original block. \\

\noindent
Execution is different for each type of block. A NormalBlock simply executes
all its executable items in order. A ConditionalBlock executes its condition
statement first. If the result is non-zero, it then executes all its executable
items. \\

\noindent
The Line class is responsible for taking a collection of tokens and finding
the meaning of the tokens. In other words, it converts a collection of tokens
into a postfix stack of operators and operands. The line class is able to find
any errors that the Tokenizer may have missed. A Line executes by processing
its postfix stack in a well known manner
(see http://academics.tjhsst.edu/compsci/apcs/january/node3.html).

\section{Usage}

\subsection{Using RTC as a user-defined initial condition}

To create a user-defined initial condition, somewhere in the definition of
your physics, you'll need to do the following:

{\ttfamily \begin{verbatim}
user defined initial condition, varName [, block n m z ...]
"
 Function goes here.
"
end
\end{verbatim} }

\noindent
The above will initial the variable corresponing to varName for blocks n,m,z.
If the block specification is left out, all blocks will be affected.
Use coord[0,1,2] to access coordinates of the object whose initial condition
is being defined. coord[0] gives the x-coord, coord[1] gives the y-coord, and
coord[2] gives the z-coord. For 2d, only coord[0,1] will work. Store the
values for varName in the field array. If varName is scalar, use only field[0].
If varName is 3-dimensional, use field[0,1,2]. In other works, the coord array
is your input and the field array is your output.\\

\noindent
An example may be the best way to illustrate the usefulness of runtime
compiled functions:

{\ttfamily \begin{verbatim}
user defined initial condition, density, block 5
"
  field[0] = 100.0;
  if(coord[0] > 0.0){
    double distance = sqrt ((coord[0]^2) + (coord[1]^2) + (coord[2]^2));
    field[0] = field[0] + distance;
  }
"
end
\end{verbatim} }
\noindent
The above function first sets \texttt{field[0]} to a value of 100.0, then in
the region of the domain where \texttt{coord[0]} exceeds 0.0 field is
incremented by the distance from the origin. In this context \texttt{coord[0]}
is the \texttt{X} coordinate, and the resulting evaluation will calculate a
constant value of \texttt{field[0]} in the negative half plane and a
monotonically increasing value of \texttt{field[0]} in the positive half plane.\\

\noindent
See section 2.3.7 for additional examples.

\subsection{Using RTC as a library}

This section will describe how to use RTC in general. \\

\noindent
There is only one Function constructor and it takes a string and an integer
argument. The string is the name of the Function, the integer is the number of
arguments the Function will take. For example:
{\ttfamily \begin{verbatim}
Function sampleFunc("sample", 2);
\end{verbatim} }
\noindent
This sets up the Function sampleFunc to take two arguments.\\

\noindent
Next, you must specify the type of each argument. You do this by calling the
addVar(string type, string name) method. It adds a variable of type type and
name name to the argument list for the function. For example:
{\ttfamily \begin{verbatim}
sampleFunc.addVar("int", "myint");
\end{verbatim} }
\noindent
This adds an integer named myint to the arg list for sampleFunc. It will return
false if it fails. Note: having whitespace in these arguments will cause an
error to be generated. \\

\noindent
After you have specified the type and name for all of the arguments, you can
compile the body of the code. You do this by calling the addBody(string body)
method. It returns false if it fails. For example:
{\ttfamily \begin{verbatim}
sampleFunc.addBody(s);
\end{verbatim} }
\noindent
This compiles the program stored in the string s. \\

\noindent
After you have compiled, at some point, much later in the program if you wish,
you can fill the values or addresses of the arguments for the function. You
do this with the varValueFill, varAddrFill, and arrayAddrFill methods. These
methods return false if they fail. For example:
{\ttfamily \begin{verbatim}
sampleFunc.varValueFill(0, 10);
\end{verbatim} }
\noindent
This call would fill the first argument in the argument list with the value
10. Some things to note: notice that the first argument passed to varValueFill
is an integer. This integer is an index to the argument list of sampleFunc.
Also, varValueFill is only for arguments that you want to pass by value. If
you want to pass by-reference, use the varAddrFill method. If you want to pass
in an array, use the arrayAddrFill method. \\

\noindent
Once you have added all the arguments, compiled the body, and filled all the
arguments, you are ready to execute the function. You do this using the
execute() method. For example:
{\ttfamily \begin{verbatim}
sampleFunc.execute();
\end{verbatim} }
\noindent
This will run sampleFunc(). \\

\noindent
The following is a complete example use of the program.
Note: In order to keep the code clean, I avoid checking for errors, but I
highly recommend checking for errors after calling methods.
{\ttfamily \begin{verbatim}
//BEGIN
	Function factorial("factorial", 2);
	factorial.addVar("int", "fac");
	factorial.addVar("int[]", "intarray");

	//...

	int fac = 1;
	int* addr = &fac;
	int arraySize = 5;

	int* intarray = new int[arraySize];
	for (int i = 0; i < 5; ++i)
	    intarray[i] = i+1;

	factorial.varAddrFill(0, addr);
	factorial.arrayAddrFill(1, intarray, arraySize);

	string s = "for (int i = 0; i < 5; i=i+1) { \
		 	fac = fac * intarray[i]; \
		    }";

	factorial.addBody(s);

	factorial.execute();

	cout << "Factorial of 5: " << fac << endl;
//END
\end{verbatim} }


%% NOTE: this file is shared directly with the Alegra Users Manual
%% NOTE: this file is shared directly between the Alegra Users Manual and
%%       the RTC writeup in alegra/toolkit/rtcompiler/writeup.tex

\subsection{The RTC language}

The RTC language can be thought of as a small subset of the
C language with a couple minor modifications. 

\subsubsection{Operators}

The RTC language has the following operators that work exactly as they do in C
and have the same precedence as they do in C:
\begin{itemize}
  \item $+$  Addition
  \item $-$  Subtration
  \item $-$  Negation
  \item $*$  Multiplication
  \item $/$  Division
  \item $==$ Equality
  \item $>$  Greater than
  \item $<$  Less than
  \item $>=$ Greater than or equal to
  \item $<=$ Less than or equal to
  \item $=$  Assignment
  \item $||$ Logical or
  \item $\&\&$ Logical and
  \item $!=$ Inequality
  \item $\%$ Modulo
  \item $!$  Logical not
\end{itemize}

\noindent The following operators do not occur in the C language, but
I added them into the RTC language for convenience:

\begin{itemize}
  \item \begin{verbatim}^ Exponentiation \end{verbatim}
\end{itemize}


\subsubsection{Control flow}

The RTC language has the following control flow statements:

\begin{itemize}
  \item for( expr ; expr ; expr ) \{ ... \}
  \item while( expr )  \{ ... \}
  \item if (expr) \{...\}
  \item else if (expr) \{...\}
  \item else \{...\}
\end{itemize}

\noindent These control flow statements work exactly as they do in C
except that the code blocks following a control flow statement 
\textbf{MUST} be enclosed within braces even if the block only consists of 
one line.

\subsubsection{Line Structure}

The line structure in the RTC language is the same as that of C. Expressions
end with a semicolon unless they are inside a control flow statement.

\subsubsection{Variables}

Declaring scalar variables in RTC is done exactly as it is done in C except 
that we are limited to the following types:

\begin{itemize}
  \item int
  \item float
  \item double
  \item char
\end{itemize}

\noindent For scalars, variables can be declared and assigned all at once. Both of the 
following approaches will work:

{\ttfamily \begin{verbatim}
int myVar = 9;

OR

int myVar;
myVar = 9;
\end{verbatim}
}

\noindent Arrays work a little differently in RTC than they do in C. There are
no \emph{new} or \emph{malloc} operators, so I have allowed the user to 
declare dynamically sized arrays in the same manner as statically sized 
arrays. Also, in C you can initialize all the values of an array at once by
putting the values within braces. In the RTC language, this is not supported.
Users will have to loop through the array and assign the values one by one.
For example:

{\ttfamily \begin{verbatim}
LEGAL:
   int ia[x*y];  //Note: in C this would not be legal for non-const x,y
   int ia2[3];

NOT LEGAL:
   int ia[3] = {1, 2, 3};
\end{verbatim}
}

\noindent Indexing arrays can be done using the index operator:
array[expr] = ...;

\noindent Bounds checking is done at run time. If you go past the end of an 
array, it will dump an error to stdout. 

\subsubsection{Math}

The following math.h functions are available in RTC:

\begin{itemize}
  \item asin(arg)  : returns the arc sine of arg
  \item acos(arg)  : returns the arc cosine of arg
  \item atan(arg)  : returns the arc tangent of arg
  \item atan2(y, x): returns the arc tangent of y/x
  \item sin(arg)   : returns the sine of arg
  \item cos(arg)   : returns the cosine of arg
  \item tan(arg)   : returns the tangent of arg
  \item sqrt(arg)  : returns the square root of arg
  \item exp(arg)   : returns the natural logarithm base e raised to the arg 
                     power
  \item sinh(arg)  : returns the hyperbolic sine of arg
  \item cosh(arg)  : returns the hyperbolic cosine of arg
  \item tanh(arg)  : returns the hyperbolic tangent of arg
  \item log(arg)   : returns the natural logarithm for arg
  \item log10(arg) : returns the base 10 logarithm for arg
  \item rand(arg)  : returns a system-generated random integer between 0 and RAND\_MAX using seed arg
  \item fabs(arg)  : returns the absolute value of arg
  \item pow(b, e)  : returns b to the e power 
                     (Note: you can use the Exponentiation operator instead)
  \item j0(arg)    : Bessel function of order zero
  \item j1(arg)    : Bessel function of order one
  \item i0(arg)    : Modified Bessel function of order zero
  \item i1(arg)    : Modified Bessel function of order one
  \item erf(arg)   : Error function 
  \item erfc(arg)  : Complementary error function  (1.0 - erf(x))
  \item gamma(arg)  : returns $\Gamma(arg)$
\end{itemize}

\subsubsection{Strings}

The user will now have the ability to pass quoted strings as arguments to 
functions. Note: you might have to escape-out the double quotes so that they
do not confuse the input-file parser. See printf section below for an example.

\subsubsection{Printf}

RTC now has much better output capability in its new printf method. The
new printf method is called just like its C counterpart. This first argument
is a quoted character string. This string will can contain the \% symbol 
which will tell RTC to output the corresponding argument. The only difference
between RTC's printf and C's printf is that in RTC's version, you do not have
to provide a type character after the \%. For example, inside your RTC method
you could do the following:

{\ttfamily \begin{verbatim}
printf(\"One:% Two:% Three:% \", 5-4, 2.0e0, 'c');\n\
\end{verbatim} }

\noindent
Which would generate this output: One:1 Two:2 Three:c

\subsubsection{Comments}

The traditional C-comment mechanism is now available for use inside RTC
functions. Use /* to begin a comment and */ to end the comment.

\subsubsection{Unsupported Features}

Implementing the entire C-language was well beyond the intend of RTC. We took
the liberty of cutting corners or leaving out features if we felt they were 
either too difficult or did not add enough value. The following is a list
of common C features that are unsupported in RTC:
\begin{itemize}
  \item There are no $++$ or $--$ operators. Use $i = i + 1$ instead of $++i$
  \item Structs
  \item Pointers
  \item Instant array initialization: int array[5] = {1,2,3,4,5};
  \item Case statements
  \item Casting
  \item Labels and gotos
  \item Function definition/declaration
  \item stdio
  \item Keywords: break, continue, const, enum, register, return, sizeof,
    typedef, union, volatile, static.
\end{itemize} 

\subsubsection{Examples}

I am going to use the following series of examples to try to illustrate the
use of the RTC language within the context of some simple user defined initial 
conditions.

\noindent For all user defined initial conditions, you can assume that you have
the following variables: 

\begin{itemize}
  \item coord - An array of coordinates. Use an index of zero to get the x 
                coordinate, an index of one to get the y coordinate, and an
                index of two to get the z coordinate (Available in 3D only).
  \item field - This is the means by which you can return the results of your
                function. The variable (ex: density, velocity) you specified 
                above the function is set according to the values of the field
                array. If you are setting a scalar variable, like density, you
                will only want to set field[0].
\end{itemize}

\subsubsection*{Example 1}

{\ttfamily \begin{verbatim}
  USER DEFINED INITIAL CONDITION, DENSITY
    "field[0] = 5000.0 + (1.0 / (coord[0] + atan2(2,3)));"
  END
\end{verbatim} }

\noindent In this example, we are setting the density of every element
equal to 5000 plus one over the x coordinate of the element plus the
arc tangent of 2 and 3.

\subsubsection*{Example 2}

{\ttfamily \begin{verbatim}
  USER DEFINED INITIAL CONDITION, DENSITY, BLOCK 5
  "
    double sum = coord[0] + coord[1] + coord[2];
    field[0] = sum / 0.0001;
  "
  END
\end{verbatim} }

\noindent In this example, we set the initial density of elements in
block 5 to the sum of the x, y, and z coordinates divided by 0.0001;

\subsubsection*{Example 3}

{\ttfamily \begin{verbatim}
  USER DEFINED INITIAL CONDITION, DENSITY, BLOCK 5
  "
    double newarray[10];
    field[0] = 0;
    for (int i = 0; i < 10; i = i + 1) {
      newarray[i] = -sin(-i*2) + 2;
    }
    for (int i = 0; i < 10; i = i + 1) {
      field[0] = field[0] + newarray[i % 10];
    }
  "
  END
\end{verbatim} }

\noindent This example shows how to use for-loops and arrays in the
RTC language. The density of the elements in block 5 is being set to
$\sum_{i=0}^9 (-sin(-i*2) + 2)$

\subsubsection*{Example 4}

{\ttfamily \begin{verbatim}
  USER DEFINED INITIAL CONDITION, VELOCITY, BLOCK 5 10
  "
     if ( fabs(coord[0]) < 1.10 ) {
       field[0] = 0.;
       field[1] = 0.;
       field[2] = 0.;
     }
     else {
       field[0] = 100.;
       field[1] = 200.;
       field[2] = 300.;
     }
  "
  END
\end{verbatim} }

\noindent In this example, we are taking the absolute value of the
x-coordinate of the element. If this value is less that 1.10, we set
velocity to zero in each direction, otherwise we set velocity to
100,200,300.

\noindent Notice in this example we are setting velocty, which is not
a scalar value, so we must assign to several indices of the field
array.

\subsection{Using RTC and APREPRO}

Frequently \textsc{alegra} users will place \texttt{aprepro} constructs into
their input decks and then preprocess the input deck with \texttt{aprepro} by
issuing the command:

\noindent
\texttt{Alegra -a runid.inp}

A problem may exist with curly braces, \{ and \}, in the runtime compiler coding
as in the above examples.  When the input deck is sent through \texttt{aprepro},
the preprocessor will evaluate expressions in curly braces, and the braces
will not appear in the processed input deck read by \textsc{alegra}.
This will cause an error when the runtime compiler processes the coding.

\noindent
There are two solutions:
\begin{enumerate}

\item Place the following lines before after the runtime compiler coding
      so that \texttt{aprepro} will copy the input lines to the output
      exactly as they are written:

{\ttfamily \begin{verbatim}
${VERBATIM(ON)}
    ... runtime compiler coding ...
${VERBATIM(OFF)}
\end{verbatim}
}

\item Omit the verbatim commands, but put the curly braces into string expressions
      that will be processed by \texttt{aprepro}.  Make the following substitutions:

{\ttfamily \begin{verbatim}
{   ->   {"{"}
}   ->   {"}"}
\end{verbatim}
}

The outer pair of opening and closing braces will be processed by \texttt{aprepro},
but the inner brace in quotes will be sent as a string to the output deck.

\item You can also place a backslash infront of your curly braces. This will
      tell APREPRO to ignore the curly brace. The RTC parser knows to ignore
      the backslash but not the curly brace. This method will work regardless
      of whether aprepro is run on the input deck or not.

\end{enumerate}

\noindent
For instance, we could write,
%
{\ttfamily \begin{verbatim}
${VERBATIM(ON)}
user defined initial condition, density, block 5
" 
  field[0] = 100.0;
  if(coord[0] > 0.0){
    double distance = sqrt ((coord[0]^2) + (coord[1]^2) + (coord[2]^2));
    field[0] = field[0] + distance;
  }
"
${VERBATIM(OFF)}
\end{verbatim}
}

\noindent
or it could be written as,
%
{\ttfamily \begin{verbatim}
user defined initial condition, density, block 5
" 
  field[0] = 100.0;
  if(coord[0] > 0.0) {"{"}
    double distance = sqrt ((coord[0]^2) + (coord[1]^2) + (coord[2]^2));
    field[0] = field[0] + distance;
   {"}"}
"
\end{verbatim}
}

\noindent
or it could be written as,
%
{\ttfamily \begin{verbatim}
user defined initial condition, density, block 5
" 
  field[0] = 100.0;
  if(coord[0] > 0.0) \{
    double distance = sqrt ((coord[0]^2) + (coord[1]^2) + (coord[2]^2));
    field[0] = field[0] + distance;
  \}
"
\end{verbatim}
}



\subsection{Adding Runtime-bound functions}

Runtime-bound functions are essentially a library of functions that are
available to RTC functions. This library can be expanded or two ways. \\

\noindent
First, one could edit the Registrar.cc and RTC\_RegistrarRTC.hh files
directly. In RTC\_RegistrarRTC.hh you would declare a new class that is a
sub-class of the RTBoundFunc class. You need the contructor of this class
to contruct the parent class (RTBoundFunc) with the name of the function
and the number of arguments the function will expect. An optional third
argument specifies whether this function should be optimized. The compiler
will try to optimize functions that have constands for all their arguments.
For example, $\sin(4.5)$, would be replaced at compile time with the result
of this call. However, certain functions, like print(x) should not be
optimized. This is what the optimization argument is for. Here is an example
of what the declaration of a new Runtime-bound function should look like:
{\ttfamily \begin{verbatim}
class Example : public RTBoundFunc
{
  public:
    Example() : RTBoundFunc("my_func", 2) {}

    double execute(Value**);
};
\end{verbatim} }
\noindent
Once you have your new Runtime-bound declared, you must provide the definition
of its execute method and register it with the Registrar class, the class that
contains all callable Runtime-bound functions. To achieve this, open the
Registrar.cc file. Go to the Registrar::setup\_standard\_functions() method.
In this method, somewhere inside the ISINIT if-block, add the following
line:
{\ttfamily \begin{verbatim}
FUNCTIONS[Example().name()] = new Example();
\end{verbatim} }
\noindent
Providing a definition for the execute method is tricky. The input argument
to this method is an array of Values. The value interface is located in
the file RTC\_ValueRTC.hh. You are allowed to ask these values for their type
and can set or ask for their values. For example, say we wanted our Example
rtbound function to assume its first argument is an array and to fill this
array with the value of the second argument and to return the value of the
second argument. Here is what that definition would look like:
{\ttfamily \begin{verbatim}
double Example::execute(Value** args)
{
  assert(args[0]->getObjectType() == ArrayVarOT);

  //fills the array in args[0] with the value of args[1]
  for (int i = 0; i < args[0]->getSize(); ++i)
    args[0]->setArrayValue(args[1]->getValue(), i);

  return args[1]->getValue();
}
\end{verbatim} }
\noindent
Notice that it might be wise to do some type checking at the top of the
method. As far as I could tell, there was no simple way of doing argument
type-checking at compile time. \\

\noindent
The second way to expand the library is create a new RTBoundFunc subclass in
your own source file. During initialization time, you can call
Register::register\_function(...) and add your new RTBoundFunc without having
to modify the RTC code. This approach is probably better if your RTBoundFunc
would cause dependencies if it relied on special function calls.

\noindent
Once you've registered your RTBoundFunc, you should be able to call it just
like any other function ($\sin$, $\cos$, etc).

\noindent
There is a new type of runtime-bound function available to users: functions
without a fixed signature, like printf. In order to have your RTBoundFunc be
a variable signature function, there is an optional fourth argument for the
constructor. See the implementation of the class Printf in RTC\_RegistrarRTC.hh and
Registrar.cc for an example.

\subsection{Recent Changes}

\begin{itemize}
  \item printf - See printf subsection (Added 3/8/06)
  \item comments - See comments subsection (Added 3/8/06)
  \item variable signature runtime-bound functions - See end of runtime-bound
        function section. (Added 3/8/06)
\end{itemize}

\subsection{Troubleshooting}

The easiest way to figure out why your function is not running is to look at
the error messages if any are generated. Most error messages will tell you the
line of the problem. In general, most of the methods on a Function object that
the user would be calling will return a bool. If the bool is false, that means
there was an error. Error messages can be retrieved by calling the getErrors()
method. So, the general form of your usage of the program should resemble the
following:
{\ttfamily \begin{verbatim}
if (!someFunctionObject.someMethod())
  cout << functionObject.getErrors();
\end{verbatim} }

\noindent
If you are really having trouble with an error or bug, feel free to email me
for help at: \begin{verbatim} jgfouca@sandia.gov \end{verbatim}

\end{document}
