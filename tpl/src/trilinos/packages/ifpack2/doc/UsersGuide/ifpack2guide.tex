%
% $Id: SANDExampleReportNotstrict.tex,v 1.26 2009-05-01 20:59:19 rolf Exp $
%
% This is an example LaTeX file which uses the SANDreport class file.
% It shows how a SAND report should be formatted, what sections and
% elements it should contain, and how to use the SANDreport class.
% It uses the LaTeX report class, but not the strict option.
%
% Get the latest version of the class file and more at
%    http://www.cs.sandia.gov/~rolf/SANDreport
%
% This file and the SANDreport.cls file are based on information
% contained in "Guide to Preparing {SAND} Reports", Sand98-0730, edited
% by Tamara K. Locke, and the newer "Guide to Preparing SAND Reports and
% Other Communication Products", SAND2002-2068P.
% Please send corrections and suggestions for improvements to
% Rolf Riesen, Org. 9223, MS 1110, rolf@cs.sandia.gov
%
\documentclass[pdf,12pt,report]{SANDreport}
\usepackage{algpseudocode}
\usepackage{amsthm}
\usepackage{booktabs}
\usepackage{calc}
\usepackage{eso-pic}
\usepackage{fancyhdr}
\usepackage{ifthen}
\usepackage{indentfirst}
\usepackage{geometry}
\usepackage{graphicx}
\usepackage[colorlinks, bookmarksopen, %pagebackref=true, backref=page,
             linkcolor={blue},
             anchorcolor={black},
             citecolor={blue},
             filecolor={magenta},
             menucolor={blue},
             pagecolor={red},
             plainpages=false,pdfpagelabels,
             pdfauthor={Andrey Prokopenko, Chris Siefert, Jonathan J. Hu, Mark
             Hoemmen, Alicia Klinvex},
             pdftitle={Ifpack2 User's Guide},
             pdfkeywords={Ifpack2,preconditioners,guide,user},
             urlcolor={blue}]{hyperref}
\usepackage{listings}
\usepackage{mathptmx}	% Use the Postscript Times font
\usepackage{multirow}
\usepackage{pifont}
\usepackage[FIGBOTCAP,normal,bf,tight]{subfigure}
\usepackage{tabularx}
\usepackage{verbatim}
\usepackage{xspace}
\usepackage{flowchart} % also loads tikz
\usepackage{algorithm}
\usetikzlibrary{arrows}

%\usepackage{draftwatermark}
%\SetWatermarkScale{.5}

\algrenewcommand{\algorithmiccomment}[1]{\hskip3em // #1}

%%%%%%%%%%%%%%%%%%%%%%%%%%%%%%%%%%%%%%%%%%%%%%%%%%%%%%%%%%%%%%%%%%%%%%%%%%%%%%%%%%%%%%%%%%%%%%%%%%%%%%%%%%%%%%%%%%%%%
% Want larger todonotes on margins?
% First, use package showframes to show the frames
% Then, adjust the geometry
% NOTE: this must be removed in the final version
% \usepackage{showframe}
% \setlength{\marginparwidth}{3.5cm}

% Add disable to todonotes options to disable all TODO notes without removing them
% \usepackage[colorinlistoftodos,prependcaption,textsize=small]{todonotes}

% \usepackage{xargs}
% \usepackage{soul}
% \newcommandx{\fix}     [3][1=]{\todo[linecolor=red,backgroundcolor=red!25,bordercolor=red,#1]{\textbf{#2: }#3}}
% \newcommandx{\unsure}  [3][1=]{\todo[linecolor=green,backgroundcolor=green!25,bordercolor=green,#1]{\textbf{#2: }#3}}
% \newcommandx{\improve} [3][1=]{\todo[linecolor=blue,backgroundcolor=blue!25,bordercolor=blue,#1]{\textbf{#2: }#3}}
% \newcommandx{\info}    [3][1=]{\todo[linecolor=gray,backgroundcolor=gray!25,bordercolor=gray,#1]{\textbf{#2: }#3}}
% \newcommandx{\fixhl}   [2]    {\texthl{#1}\fix{#2}}
%%%%%%%%%%%%%%%%%%%%%%%%%%%%%%%%%%%%%%%%%%%%%%%%%%%%%%%%%%%%%%%%%%%%%%%%%%%%%%%%%%%%%%%%%%%%%%%%%%%%%%%%%%%%%%%%%%%%%


% If you want to relax some of the SAND98-0730 requirements, use the "relax"
% option. It adds spaces and boldface in the table of contents, and does not
% force the page layout sizes.
% e.g. \documentclass[relax,12pt]{SANDreport}
%
% You can also use the "strict" option, which applies even more of the
% SAND98-0730 guidelines. It gets rid of section numbers which are often
% useful; e.g. \documentclass[strict]{SANDreport}



% ---------------------------------------------------------------------------- %
%
% Set the title, author, and date
%
\title{Ifpack2 User's Guide 1.0 \\
(Trilinos version 12.6)}

\author{
  Andrey Prokopenko \\
  Scalable Algorithms \\
  Sandia National Laboratories\\
  Mailstop 1318 \\
  P.O.~Box 5800 \\
  Albuquerque, NM 87185-1318\\
  aprokop@sandia.gov\\
  \and
  Christopher Siefert \\
  Computational Math \& Algorithms \\
  Sandia National Laboratories\\
  Mailstop 1318 \\
  P.O.~Box 5800 \\
  Albuquerque, NM 87185-1318 \\
  \and
  Jonathan J. Hu \\
  Scalable Algorithms \\
  Sandia National Laboratories\\
  Mailstop 9159 \\
  P.O.~Box 0969 \\
  Livermore, CA 94551-0969\\
  jhu@sandia.gov \\
  \and
  Mark Hoemmen \\
  Scalable Algorithms \\
  Sandia National Laboratories\\
  Mailstop 1320 \\
  P.O.~Box 5800 \\
  Albuquerque, NM 87185-1318\\
  mhoemme@sandia.gov\\
  \and
  Alicia Klinvex \\
  Scalable Algorithms \\
  Sandia National Laboratories\\
  Mailstop 1320 \\
  P.O.~Box 5800 \\
  Albuquerque, NM 87185-1318\\
  amklinv@sandia.gov\\
}

% There is a "Printed" date on the title page of a SAND report, so
% the generic \date should generally be empty.
\date{}

\newcommand{\amesos}       {\textsc{Amesos}\xspace}
\newcommand{\amesostwo}    {\textsc{Amesos2}\xspace}
\newcommand{\anasazi}      {\textsc{Anasazi}\xspace}
\newcommand{\aztecoo}      {\textsc{AztecOO}\xspace}
\newcommand{\belos}        {\textsc{Belos}\xspace}
\newcommand{\epetra}       {\textsc{Epetra}\xspace}
\newcommand{\epetraext}    {\textsc{EpetraExt}\xspace}
\newcommand{\galeri}       {\textsc{Galeri}\xspace}
\newcommand{\ifpack}       {\textsc{Ifpack}\xspace}
\newcommand{\ifpacktwo}    {\textsc{Ifpack2}\xspace}
\newcommand{\isorropia}    {\textsc{Isorropia}\xspace}
\newcommand{\loca}         {\textsc{Loca}\xspace}
\newcommand{\ml}           {\textsc{ML}\xspace}
\newcommand{\muelu}        {\textsc{MueLu}\xspace}
\newcommand{\nox}          {\textsc{NOX}\xspace}
\newcommand{\stratimikos}  {\textsc{Stratimikos}\xspace}
\newcommand{\teuchos}      {\textsc{Teuchos}\xspace}
\newcommand{\teko}         {\textsc{Teko}\xspace}
\newcommand{\thyra}        {\textsc{Thyra}\xspace}
\newcommand{\tpetra}       {\textsc{Tpetra}\xspace}
\newcommand{\trilinos}     {\textsc{Trilinos}\xspace}
\newcommand{\xpetra}       {\textsc{Xpetra}\xspace}
\newcommand{\zoltan}       {\textsc{Zoltan}\xspace}
\newcommand{\zoltantwo}    {\textsc{Zoltan2}\xspace}


\newcommand{\klu}          {\textsc{Klu}\xspace}
\newcommand{\metis}        {\textsc{Metis}\xspace}
\newcommand{\mumps}        {\textsc{Mumps}\xspace}
\newcommand{\umfpack}      {\textsc{Umfpack}\xspace}
\newcommand{\superlu}      {\textsc{SuperLU}\xspace}
\newcommand{\superludist}  {\textsc{SuperLU\_dist}\xspace}
\newcommand{\parmetis}     {\textsc{ParMetis}\xspace}
\newcommand{\paraview}     {\textsc{ParaView}\xspace}

\newcommand{\parameterlist}{\texttt{ParameterList}\xspace}

\newcommand \trilinosWeb   {trilinos.sandia.gov\xspace}

%\newcommand{\be}  {\begin{enumerate}}
%\newcommand{\ee}  {\end{enumerate}}
%\newcommand{\cba}[3]{\choicebox{\texttt{#1}}{[{\texttt #2}] #3}}
%\newcommand{\cbb}[4]{\choicebox{\texttt{#1}}{[{\texttt #2}] #4 {\bf Default:~}#3.}}
%\newcommand{\cbc}[4]{\choicebox{\texttt{\color{red}#1}}{[{\texttt #2}] #4 {\bf Default:~}#3.}}
%
%\newcommand{\comm}[2]{\bigskip
%                      \begin{tabular}{|p{4.5in}|}\hline
%                      \multicolumn{1}{|c|}{{\bf Comment by #1}}\\ \hline
%                      #2\\ \hline
%                      \end{tabular}\\
%                      \bigskip
%                     }


\newtheorem*{mycomment}{\ding{42}}
\newtheoremstyle{plain}
  {\topsep}   % ABOVESPACE
  {\topsep}   % BELOWSPACE
  {\normalfont}  % BODYFONT
  {0pt}       % INDENT (empty value is the same as 0pt)
  {\bfseries} % HEADFONT
  {}         % HEADPUNCT
  {5pt plus 1pt minus 1pt} % HEADSPACE
  {}          % CUSTOM-HEAD-SPEC

% further declarations and additional commands
\definecolor{hellgelb}{rgb}{1,1,0.8}   % background color for C++ listings
\definecolor{darkgreen}{rgb}{0.0, 0.2, 0.13}
%\definecolor{hellrot}{HTML}{FFA4C2}    % background color for xml files

% settings for listings package
\lstset{
  backgroundcolor=\color{hellgelb},
  basicstyle=\ttfamily\small,
  breakautoindent=true,
  breaklines=true,
  captionpos=b,
  columns=flexible,
  commentstyle=\color{darkgreen},
  extendedchars=true,
  float=hbp,
  frame=single,
  identifierstyle=\color{black},
  keywordstyle=\color{blue},
  numbers=none,
  numberstyle=\tiny,
  showspaces=false,
  showstringspaces=false,
  stringstyle=\color{purple},
  tabsize=2,
}


% ---------------------------------------------------------------------------- %
% Set some things we need for SAND reports. These are mandatory
%
\SANDnum{SAND2016-5338}
\SANDprintDate{June 2016}
\SANDauthor{Andrey Prokopenko, Christopher M. Siefert, Jonathan J. Hu, \\Mark
Hoemmen, Alicia Klinvex}


% ---------------------------------------------------------------------------- %
% Include the markings required for your SAND report. The default is "Unlimited
% Release". You may have to edit the file included here, or create your own
% (see the examples provided).
%
% \include{MarkUR} % Not needed for unlimted release reports


% ---------------------------------------------------------------------------- %
% The following definition does not have a default value and will not
% print anything, if not defined
%
%\SANDsupersed{SAND1901-0001}{January 1901}
%\input{MarkOUO}


% ---------------------------------------------------------------------------- %
%
% Start the document
%
\begin{document}

    \maketitle

    % ------------------------------------------------------------------------ %
    % An Abstract is required for SAND reports
    %
    \begin{abstract}
	This is the definitive user manual for the \ifpacktwo{} package in the Trilinos
project.  \ifpacktwo{} provides implementations of iterative algorithms (e.g.,
Jacobi, SOR, additive Schwarz) and processor-based incomplete factorizations.
\ifpacktwo{} is part of the Trilinos \tpetra{} solver stack, is templated on index,
scalar, and node types, and leverages node-level parallelism indirectly through
its use of \tpetra{} kernels.  \ifpacktwo{} can be used to solve to matrix systems
with greater than 2 billion rows (using 64-bit indices). \textit{Any options not documented in this
  manual should be considered strictly experimental}.

    \end{abstract}


    % ------------------------------------------------------------------------ %
    % An Acknowledgement section is optional but important, if someone made
    % contributions or helped beyond the normal part of a work assignment.
    % Use \section* since we don't want it in the table of context
    %
    \clearpage
    \chapter*{Acknowledgment}
	Many people have helped develop \ifpacktwo{}, and we would like to acknowledge
their contributions here: Ross Bartlett, Tom Benson, Erik Boman, Joshua Booth,
Julian Cortial, Kevin Deweese, Jeremie Gaidamour, Paul Lin, Travis Fisher, Sarah
Osborn, Eric Phipps, and Paul Tsuji. Finally, Alan Williams did the original
port from \ifpack{} and was the original lead developer of \ifpacktwo{}.


    % ------------------------------------------------------------------------ %
    % The table of contents and list of figures and tables
    % Comment out \listoffigures and \listoftables if there are no
    % figures or tables. Make sure this starts on an odd numbered page
    %
    \cleardoublepage		% TOC needs to start on an odd page
    \tableofcontents
    \listoffigures
    \listoftables


    % ---------------------------------------------------------------------- %
    % An optional preface or Foreword
    %\clearpage
    %\chapter*{Preface}
    %\addcontentsline{toc}{chapter}{Preface}
	%\input{CommonPreface}


    % ---------------------------------------------------------------------- %
    % An optional executive summary
    %\clearpage
    %\chapter*{Summary}
    %\addcontentsline{toc}{chapter}{Summary}
	%\input{CommonSummary}


    % ---------------------------------------------------------------------- %
    % An optional glossary. We don't want it to be numbered
    %\clearpage
    %\chapter*{Nomenclature}
    %\addcontentsline{toc}{chapter}{Nomenclature}
    %\begin{description}
	%\item[dry spell]
	%    using a dry erase marker to spell words
	%\item[dry wall]
	%    the writing on the wall
	%\item[dry humor]
	%    when people just do not understand
	%\item[DRY]
	%    Don't Repeat Yourself
    %\end{description}


    % ---------------------------------------------------------------------- %
    % This is where the body of the report begins; usually with an Introduction
    %
    \SANDmain		% Start the main part of the report

    %-----------------------------%
    % \chapter{Introduction}\label{sec:introduction}
    %-----------------------------%
    % This guide gives an overview of \muelu{}'s capabilities.  If you are looking for
a tutorial, please refer to the \muelu{} tutorial in \verb!muelu/doc/Tutorial!
(see also~\cite{MueLuTutorial}). New users should start with~\S\ref{sec:getting
started}. It strives to give the new user all the information he/she might need
to begin using \muelu{} quickly. Users interested in performance, especially in
parallel context, should refer to~\S\ref{sec:performance}.  Users looking for a
particular option should consult~\S\ref{sec:options}, containing a complete set of
supported options in \muelu{}. \\

\noindent
If you find any errors or omissions in this guide, have comments or suggestions,
or would like to contribute to \muelu{} development, please contact the \muelu{}
\href{mailto:muelu-users@software.sandia.gov}{users list}, or
\href{mailto:muelu-developers@software.sandia.gov}{developers list}.


    %-----------------------------%
    \chapter{Getting Started}\label{sec:getting started}
    This section is meant to get you using \ifpacktwo{} as quickly as possible.
\S\ref{sec:overview} gives a brief overview of \ifpacktwo{}.
\S\ref{sec:configuration_and_build} lists \ifpacktwo{}'s dependencies on other
\trilinos{} libraries and provides a sample cmake configuration line. Finally,
some examples of code are given in~\S\ref{sec:examples in code}.

\section{Overview of \ifpacktwo{}}
\label{sec:overview}
\ifpacktwo{} is a C++ linear solver library in the \trilinos{} project~\cite{Heroux2012}.
It originally began as a migration of \ifpack{} package capabilities to a new linear
algebra stack. While it retains some commonalities with the original package, it
has since diverged significantly from it and should be treated as completely
independent package.

\ifpacktwo{} only works with \tpetra{}~\cite{TpetraURL} matrix,
vector, and map types. Like \tpetra{}, it allows for different ordinal
(index) and scalar types. \ifpacktwo{} was designed to be efficient on a wide
range of computer architectures, from workstations to supercomputers~\cite{Lin2014}.
It relies on the ``MPI+X" principle, where ``X'' can be threading or
CUDA\@. The ``X'' portion, node-level parallelism, is controlled by a node
template type. Users should refer to \tpetra{}'s documentation for information
about node and device types.

\ifpacktwo provides a number of different solvers, including
\begin{itemize}
  \item Jacobi, Gauss-Seidel, polynomial, distributed relaxation;
  \item domain decomposition solvers;
  \item incomplete factorizations.
\end{itemize}
This list of solvers is not exhaustive. Instead, references for further
information are provided throughout the text. There are many excellent
references for iterative methods, including~\cite{Saad2003}.

Complete information on available capabilities and options can be found
in~\S\ref{sec:options}.

\section{Configuration and Build}\label{sec:configuration_and_build}

\ifpacktwo{} requires a C++11 compatible compiler for compilation. The
minimum required version of compilers are GCC (4.7.2 and later),
Intel (13 and later), and clang (3.5 and later).

\subsection{Dependencies}

Table~\ref{tab:dependencies} enumerates the dependencies of \ifpacktwo. Certain
dependencies are optional, whereas others are required.  Furthermore,
\ifpacktwo's tests depend on certain libraries that are not required if you only
want to link against the \ifpacktwo library and do not want to compile its
tests. Additionally, some functionality in \ifpacktwo{} may depend on other
Trilinos packages (for instance, \amesostwo{}) that may require additional
dependencies. We refer to the documentation of those packages for a full list of
dependencies.

\begin{table}[ht]
  \centering
  \begin{tabular}{p{3.5cm} c c c c}
    \toprule
    \multirow{2}{*}{Dependency} & \multicolumn{2}{c}{Library} & \multicolumn{2}{c}{Testing} \\
    \cmidrule(r){2-3} \cmidrule(l){4-5} & Required & Optional & Required & Optional  \\
    \midrule
    % \belos                       & $\times$ &          & $\times$ & \\
    \teuchos                     & $\times$ &          & $\times$ & \\
    \tpetra                      & $\times$ &          & $\times$ & \\
    \tpetrakernels               & $\times$ &          &          & \\
    \amesostwo                   &          & $\times$ &          & $\times$  \\
    \galeri                      &          &          &          & $\times$  \\
    \xpetra                      &          & $\times$ &          & $\times$  \\
    \zoltantwo                   &          & $\times$ &          & $\times$  \\
    \textsc{ThyraTpetraAdapters} &          & $\times$ &          & \\
    \textsc{ShyLUBasker}         &          & $\times$ &          & $\times$ \\
    \textsc{ShyLUHTS}            &          & $\times$ &          & $\times$ \\
    \midrule
    % BLAS                         & $\times$ &          & $\times$ & \\
    % LAPACK                       & $\times$ &          & $\times$ & \\
    MPI                          &          & $\times$ &          & $\times$  \\
    % Cholmod                      &          & $\times$ &          & $\times$  \\
    % SuperLU 4.3                  &          & $\times$ &          & $\times$  \\
    % QD                           &          & $\times$ &          & $\times$  \\
    \bottomrule
  \end{tabular}
  \caption{\label{tab:dependencies}\ifpacktwo{}'s required and optional dependencies,
    subdivided by whether a dependency is that of the \ifpacktwo{}{} library itself
    (\textit{Library}), or of some \ifpacktwo{}{} test (\textit{Testing}). }
\end{table}

\amesostwo and \superlu are necessary if you want to use either a sparse direct
solve or ILUTP as a subdomain solve in processor-based domain decomposition.
\zoltantwo and \xpetra are necessary if you want to reorder a matrix (e.g.,
reverse Cuthill McKee).

\subsection{Configuration}
The preferred way to configure and build \ifpacktwo{} is to do that outside of the source directory.
Here we provide a sample configure script that will enable \ifpacktwo{} and all of its optional dependencies:
\begin{lstlisting}
  export TRILINOS_HOME=/path/to/your/Trilinos/source/directory
  cmake -D BUILD_SHARED_LIBS:BOOL=ON \
        -D CMAKE_BUILD_TYPE:STRING="RELEASE" \
        -D CMAKE_CXX_FLAGS:STRING="-g" \
        -D Trilinos_ENABLE_EXPLICIT_INSTANTIATION:BOOL=ON \
        -D Trilinos_ENABLE_TESTS:BOOL=OFF \
        -D Trilinos_ENABLE_EXAMPLES:BOOL=OFF \
        -D Trilinos_ENABLE_Ifpack2:BOOL=ON \
        -D Ifpack2_ENABLE_TESTS:STRING=ON \
        -D Ifpack2_ENABLE_EXAMPLES:STRING=ON \
        -D TPL_ENABLE_BLAS:BOOL=ON \
        -D TPL_ENABLE_MPI:BOOL=ON \
        ${TRILINOS_HOME}
\end{lstlisting}

\noindent
More configure examples can be found in \texttt{Trilinos/sampleScripts}.
For more information on configuring, see the \trilinos Cmake Quickstart guide \cite{TrilinosCmakeQuickStart}.

\section{Interface to \ifpacktwo{} methods}
All \ifpacktwo operators inherit from the base class
\texttt{Ifpack2::Preconditioner}. This in turn inherits from
\texttt{Tpetra::Operator}. Thus, you may use an \ifpacktwo operator anywhere
that a \texttt{Tpetra::Operator} works. For example, you may use \ifpacktwo operators
directly as preconditioners in \trilinos' \belos package of iterative solvers.

You may either create an \ifpacktwo operator directly, by using the class and
options that you want, or by using \texttt{Ifpack2::Factory}. Some of
\ifpacktwo preconditioners only accept a \texttt{Tpetra::\\CrsMatrix} instance as
input, while others also may accept a \texttt{Tpetra::RowMatrix} (the base class
of \texttt{Tpetra::CrsMatrix}). They will decide at run time whether the input
\texttt{Tpetra::RowMatrix} is an instance of the right subclass.

\texttt{Ifpack2::Preconditioner} includes the following methods:
\begin{itemize}
  \item \texttt{initialize()}

    Performs all operations based on the graph of the matrix (without
    considering the numerical values).

  \item \texttt{compute()}

    Computes everything required to apply the preconditioner, using the matrix's
    values.

  \item \texttt{apply()}

    Applies or ``solves with'' the preconditioner.
\end{itemize}
Every time that \texttt{initialize()} is called, the object destroys all the
previously allocated information, and reinitializes the preconditioner. Every
time \texttt{compute()} is called, the object recomputes the actual values of the
preconditioner.

An \ifpacktwo preconditioner may also inherit from
\texttt{Ifpack2::CanChangeMatrix} class in order to express that users can
change its matrix (the matrix that it preconditions) after construction using a
\texttt{setMatrix} method.  Changing the matrix puts the preconditioner back in
an ``pre-initialized'' state.  You must first call \texttt{initialize()}, then
\texttt{compute()}, before you may call \texttt{apply()} on this preconditioner.
Depending on the implementation, it may be legal to set the matrix to null. In
that case, you may not call \texttt{initialize()} or \texttt{compute()} until
you have subsequently set a nonnull matrix.

\textbf{Warning.} If you are familiar with the \ifpack package~\cite{ifpack}, please be aware
that the behaviour of the \ifpacktwo preconditioner is different from \ifpack.
In \ifpack, the \texttt{ApplyInverse()} method applies or ``solves with'' the
preconditioner $M^{-1}$, and the \texttt{Apply()} method ``applies'' the
preconditioner $M$. In \ifpacktwo, the \texttt{apply()} method applies or
``solves with'' the preconditioner $M^{-1}$. \ifpacktwo has no method comparable
to \ifpack's \texttt{Apply()}.

\section{Example: \ifpacktwo preconditioner within \belos}\label{sec:examples in code}

The most commonly used scenario involving \ifpacktwo{} is using one of its
preconditioners preconditioners inside an iterative linear solver. In
\trilinos{}, the \belos{} package provides important Krylov subspace methods (such
as preconditioned CG and GMRES).

At this point, we assume that the reader is comfortable with \teuchos{} referenced-counted
pointers (RCPs) for memory management (an introduction to RCPs can be found
in~\cite{RCP2010}) and the \parameterlist class~\cite{TeuchosURL}.

First, we create an \ifpacktwo{} preconditioner using a provided \parameterlist
\begin{lstlisting}[language=C++]
 typedef Tpetra::CrsMatrix<Scalar, LocalOrdinal, GlobalOrdinal, Node>
   Tpetra_Operator;

 Teuchos::RCP<Tpetra_Operator> A;
 // create A here ...
 Teuchos::ParameterList paramList;
 paramList.set( "chebyshev: degree", 1 );
 paramList.set( "chebyshev: min eigenvalue", 0.5 );
 paramList.set( "chebyshev: max eigenvalue", 2.0 );
 // ...
 Ifpack2::Factory factory;
 RCP<Ifpack2::Ifpack2Preconditioner<> > ifpack2Preconditioner;
 ifpack2Preconditioner = factory.create( "CHEBYSHEV", A )
 ifpack2Preconditioner->setParameters( paramList );
 ifpack2Preconditioner->initialize();
 ifpack2Preconditioner->compute();
\end{lstlisting}

Besides the linear operator $A$, we also need an initial guess vector for the
solution $X$ and a right hand side vector $B$ for solving a linear system.
\begin{lstlisting}[language=C++]
 typedef Tpetra::Map<LocalOrdinal, GlobalOrdinal, Node> Tpetra_Map;
 typedef Tpetra::MultiVector<Scalar, LocalOrdinal, GlobalOrdinal, Node>
   Tpetra_MultiVector;

 Teuchos::RCP<const Tpetra_Map> map = A->getDomainMap();

 // create initial vector
 Teuchos::RCP<Tpetra_MultiVector> X =
   Teuchos::rcp( new Tpetra_MultiVector(map, numrhs) );

 // create right-hand side
 X->randomize();
 Teuchos::RCP<Tpetra_MultiVector> B =
   Teuchos::rcp( new Tpetra_MultiVector(map, numrhs) );
 A->apply( *X, *B );
 X->putScalar( 0.0 );
\end{lstlisting}
To generate a dummy example, the above code first declares two vectors. Then, a
right hand side vector is calculated as the matrix-vector product of a random vector
with the operator $A$. Finally, an initial guess is initialized with zeros.

Then, one can define a \texttt{Belos::LinearProblem} object where the
\texttt{ifpack2Preconditioner} is used for left preconditioning.
\begin{lstlisting}[language=C++]
 typedef Belos::LinearProblem<Scalar, Tpetra_MultiVector, Tpetra_Operator>
   Belos_LinearProblem;

 Teuchos::RCP<Belos_LinearProblem> problem =
   Teuchos::rcp( new Belos_LinearProblem( A, X, B ) );
 problem->setLeftPrec( mueLuPreconditioner );
 bool set = problem.setProblem();
\end{lstlisting}

Next, we set up a \belos{} solver using some basic parameters.
\begin{lstlisting}[language=C++]
 Teuchos::RCP<Teuchos::ParameterList> belosList =
   Teuchos::rcp(new Teuchos::ParameterList);
 belosList->set( "Block Size", 1 );
 belosList->set( "Maximum Iterations", 100 );
 belosList->set( "Convergence Tolerance", 1e-10 );
 belosList->set( "Output Frequency", 1 );
 belosList->set( "Verbosity", Belos::TimingDetails + Belos::FinalSummary );

 Belos::SolverFactory<Scalar, Tpetra_MultiVector, Tpetra_Operator> solverFactory;
 Teuchos::RCP<Belos::SolverManager<Scalar, Tpetra_MultiVector, Tpetra_Operator> >
   solver = solverFactory.create( "Block CG", belosList );
 solver->setProblem( problem );
\end{lstlisting}

Finally, we solve the system.
\begin{lstlisting}[language=C++]
 Belos::ReturnType ret = solver.solve();
\end{lstlisting}

It is often more convenient to specify the parameters as part of an XML-formatted options file.
Look in the subdirectory {\tt Trilinos/packages/ifpack2/test/belos} for examples of this.

This section is only meant to give a brief introduction on how to use
\ifpacktwo{} as a preconditioner within the \trilinos{} packages for iterative
solvers. There are other, more complicated, ways to use to work with
\ifpacktwo{}. For more information on these topics, the reader may refer to the
examples and tests in the \ifpacktwo{} source directory
(\texttt{Trilinos/packages/ifpack2}).


    %-----------------------------%
    \chapter{\ifpacktwo options}
    \label{sec:muelu_options}

In this section, we report the complete list of \muelu{} input parameters.  It
is important to notice, however, that \muelu{} relies on other \trilinos{}
packages to provide support for some of its algorithms. For instance,
\ifpack{}/\ifpacktwo{} provide standard smoothers like Jacobi, Gauss-Seidel or
Chebyshev, while \amesos{}/\amesostwo{} provide access to direct solvers. The
parameters affecting the behavior of the algorithms in those packages are
simply passed by \muelu{} to a routine from the corresponding library. Please
consult corresponding packages' documentation for a full list of supported
algorithms and corresponding parameters.

\section{Using parameters on individual levels}
Some of the parameters that affect the preconditioner can in principle be
different from level to level. By default, parameters affect all levels in
a multigrid hierarchy.

The settings on a particular levels can be changed by using level sublists.
A level sublist is a \parameterlist{} sublist with the name ``level XX'', where XX is the level number. The
parameter names in the sublist do not require any modifications. For example,
the following fragment of code
\begin{lstlisting}[language=XML]
  <ParameterList name="level 2">
    <Parameter name="smoother: type" type="string" value="CHEBYSHEV"/>
  </ParameterList>
\end{lstlisting}
changes the smoother for level 2 to be a polynomial smoother.

\section{Parameter validation}
By default, \muelu{} validates the input parameter list. A parameter that is
misspelled, is unknown, or has an incorrect value type will cause an exception to be
thrown and execution to halt.

\begin{mycomment}
Spaces are important within a parameter's name. Please separate words
by just one space, and make sure there are no leading or trailing spaces.
\end{mycomment}

The option \verb|print initial parameters| prints the initial list given to the
interpreter. The option \verb|print unused parameters| prints the list of unused
parameters.

% ==================== GENERAL ====================
\section{General options}
\label{sec:options_general}

\begin{table}[h!]
  \begin{center}
    \begin{tabular}{p{3cm} p{12cm}}
      \toprule
      Verbosity level           & Description \\
      \midrule
      \verb!none!               & No output \\
      \verb!low!                & Errors, important warnings, and some statistics \\
      \verb!medium!             & Same as \verb!low!, but with more statistics \\
      \verb!high!               & Errors, all warnings, and all statistics \\
      \verb!extreme!            & Same as \verb!high!, but also includes output from other packages (\textit{i.e.}, \zoltan{}) \\
      \bottomrule
    \end{tabular}
    \caption{Verbosity levels.}
\label{t:verbosity_types}
  \end{center}
\end{table}

\begin{table}[h!]
  \begin{center}
    \begin{tabular}{p{4.3cm} p{4.3cm} c p{4.5cm}}
      \toprule
      Problem type               & Multigrid algorithm    & Block size  & Smoother \\
      \midrule
      \verb!unknown!             & --                     & --          & -- \\
      \verb!Poisson-2D!          & Smoothed aggregation   & 1           & Chebyshev \\
      \verb!Poisson-3D!          & Smoothed aggregation   & 1           & Chebyshev \\
      \verb!Elasticity-2D!       & Smoothed aggregation   & 2           & Chebyshev \\
      \verb!Elasticity-3D!       & Smoothed aggregation   & 3           & Chebyshev \\
      \verb!ConvectionDiffusion! & Petrov-Galerkin  AMG   & 1           & Gauss-Seidel \\
      \verb!MHD!                 & Unsmoothed aggregation & --          & Additive Schwarz method with one level of overlap and ILU(0) as a subdomain solver \\
      \bottomrule
    \end{tabular}
    \caption{Supported problem types (``--'' means not set).}
\label{t:problem_types}
  \end{center}
\end{table}


\cbb{problem: type}{string}{"unknown"}{Type of problem to be solved. Possible values: see Table~\ref{t:problem_types}.}
          
\cbb{verbosity}{string}{"high"}{Control of the amount of printed information. Possible values: see Table~\ref{t:verbosity_types}.}
          
\cbb{number of equations}{int}{1}{Number of PDE equations at each grid node. Only constant block size is considered.}
          
\cbb{max levels}{int}{10}{Maximum number of levels in a hierarchy.}
          
\cbb{cycle type}{string}{"V"}{Multigrid cycle type. Possible values: "V", "W".}
          
\cbb{problem: symmetric}{bool}{true}{Symmetry of a problem. This setting affects the construction of a restrictor. If set to true, the restrictor is set to be the transpose of a prolongator. If set to false, underlying multigrid algorithm makes the decision.}
          
\cbb{xml parameter file}{string}{""}{An XML file from which to read additional
      parameters.  In case of a conflict, parameters manually set on
      the list will override parameters in the file. If the string is
      empty a file will not be read.}
          

% ==================== SMOOTHERS ====================
\section{Smoothing and coarse solver options}
\label{sec:options_smoothing}

\muelu{} relies on other \trilinos{} packages to provide level smoothers and
coarse solvers. \ifpack{} and \ifpacktwo{} provide standard smoothers (see
Table~\ref{tab:smoothers}), and \amesos{} and \amesostwo{} provide direct
solvers (see Table~\ref{tab:coarse}). Note that it is completely possible to use
any level smoother as a direct solver.

\muelu{} checks parameters \verb|smoother: * type| and \verb|coarse: type| to
determine:
\begin{itemize}
  \item what package to use (i.e., is it a smoother or a direct solver);
  \item (possibly) transform a smoother type

    \ding{42} \ifpack{} and \ifpacktwo{} use different smoother type names,
    e.g., ``point relaxation stand-alone'' vs ``RELAXATION''.  \muelu{} tries to follow
    \ifpacktwo{} notation for smoother types. Please consult \ifpacktwo{}
    documentation~\cite{Ifpack2} for more information.
\end{itemize}
The parameter lists \verb|smoother: * params| and \verb|coarse: params| are
passed directly to the corresponding package without any examination of their
content. Please consult the documentation of the corresponding packages for a list of
possible values.

By default, \muelu{} uses one sweep of symmetric Gauss-Seidel for both pre- and
post-smoothing, and SuperLU for coarse system solver.

\begin{table}[tbh]
  \begin{center}
    \begin{tabular}{p{4.0cm} p{10cm}}
      \toprule
      \texttt{smoother: type}           & \\
      \midrule
      \verb|RELAXATION|                 & Point relaxation smoothers, including
                                          Jacobi, Gauss-Seidel, symmetric Gauss-Seidel, etc. The exact
                                          smoother is chosen by specifying \texttt{relaxation: type} parameter in
                                          the \texttt{smoother: params} sublist. \\
      \verb|CHEBYSHEV|                  & Chebyshev polynomial smoother. \\
      \verb|ILUT|, \verb|RILUK|         & Local (processor-based) incomplete factorization methods. \\
      \bottomrule
    \end{tabular}
    \caption{Commonly used smoothers provided by \ifpack{}/\ifpacktwo{}. Note
    that these smoothers can also be used as coarse grid solvers.}
\label{tab:smoothers}
  \end{center}
\end{table}

\begin{table}[tbh]
  \begin{center}
    \begin{tabular}{p{4.0cm} c c p{7cm}}
      \toprule
      \texttt{coarse: type}             & \amesos{} & \amesostwo{} &  \\
      \midrule
      \verb|KLU|                        & x & & Default \amesos{} solver~\cite{klu}. \\
      \verb|KLU2|                       & & x & Default \amesostwo{} solver~\cite{amesos2_belos}. \\
      \verb|SuperLU|                    & x & x & Third-party serial sparse direct solver~\cite{Li2011}. \\
      \verb|SuperLU_dist|               & x & x & Third-party parallel sparse direct solver~\cite{Li2011}. \\
      \verb|Umfpack|                    & x & & Third-party solver~\cite{umfpack}. \\
      \verb|Mumps|                      & x & & Third-party solver~\cite{mumps}. \\
      \bottomrule
    \end{tabular}
    \caption{Commonly used direct solvers provided by \amesos{}/\amesostwo{}}
\label{tab:coarse}
  \end{center}
\end{table}


\cbb{smoother: pre or post}{string}{"both"}{Pre- and post-smoother combination. Possible values: "pre" (only pre-smoother), "post" (only post-smoother), "both" (both pre-and post-smoothers), "none" (no smoothing).}
          
\cbb{smoother: type}{string}{"RELAXATION"}{Smoother type. Possible values: see Table~\ref{tab:smoothers}.}
          
\cbb{smoother: pre type}{string}{"RELAXATION"}{Pre-smoother type. Possible values: see Table~\ref{tab:smoothers}.}
          
\cbb{smoother: post type}{string}{"RELAXATION"}{Post-smoother type. Possible values: see Table~\ref{tab:smoothers}.}
          
\cba{smoother: params}{\parameterlist}{Smoother parameters. For standard smoothers, \muelu passes them directly to the appropriate package library.}
          
\cba{smoother: pre params}{\parameterlist}{Pre-smoother parameters. For standard smoothers, \muelu passes them directly to the appropriate package library.}
          
\cba{smoother: post params}{\parameterlist}{Post-smoother parameters. For standard smoothers, \muelu passes them directly to the appropriate package library.}
          
\cbb{smoother: overlap}{int}{0}{Smoother subdomain overlap.}
          
\cbb{smoother: pre overlap}{int}{0}{Pre-smoother subdomain overlap.}
          
\cbb{smoother: post overlap}{int}{0}{Post-smoother subdomain overlap.}
          
\cbb{coarse: max size}{int}{2000}{Maximum dimension of a coarse grid. \muelu will stop coarsening once it is achieved.}
          
\cbb{coarse: type}{string}{"SuperLU"}{Coarse solver. Possible values: see Table~\ref{tab:coarse_solvers}.}
          
\cba{coarse: params}{\parameterlist}{Coarse solver parameters. \muelu passes them directly to the appropriate package library.}
          
\cbb{coarse: overlap}{int}{0}{Coarse solver subdomain overlap.}
          

% ==================== AGGREGATION ====================
\section{Aggregation options}
\label{sec:options_aggregation}

\begin{table}[h!]
  \begin{center}
    \begin{tabular}{p{5.0cm} p{10cm}}
      \toprule
      \verb!uncoupled! & Attempts to construct aggregates of optimal size ($3^d$
                         nodes in $d$ dimensions). Each process works independently, and
                         aggregates cannot span multiple processes.\\
      \verb!coupled!   & Attempts to construct aggregates of optimal size ($3^d$
                         nodes in $d$ dimensions). Aggregates are allowed to
                         cross processor boundaries. \textbf{Use carefully}. If
                         unsure, use \verb!uncoupled! instead.\\
      \verb!brick!     & Attempts to construct rectangular aggregates \\
      %\verb!METIS!     & Use graph partitioning algorithm to create aggregates,
      %                   working process-wise. Number of nodes in each aggregate
      %                   is specified with option \texttt{aggregation: max agg
      %                   size}. \\
      % \verb!ParMETIS!  & As \verb!METIS!, but partition global graph. Aggregates
                         % can span arbitrary number of processes. Specify global
                         % number of aggregates with {\tt aggregation: global
                         % number}. \\
      \bottomrule
    \end{tabular}
    \caption{Available coarsening schemes. }
\label{t:aggregation}
  \end{center}
\end{table}


\cbb{aggregation: type}{string}{"uncoupled"}{Aggregation scheme. Possible values: see Table~\ref{t:aggregation}.}
          
\cbb{aggregation: ordering}{string}{"natural"}{Node ordering strategy. Possible values: "natural" (local index order), "graph" (filtered graph breadth-first order), "random" (random local index order).}
          
\cbb{aggregation: drop scheme}{string}{"classical"}{Connectivity dropping scheme for a graph used in aggregation. Possible values: "classical", "distance laplacian".}
          
\cbb{aggregation: drop tol}{double}{0.0}{Connectivity dropping threshold for a graph used in aggregation.}
          
\cbb{aggregation: min agg size}{int}{2}{Minimum size of an aggregate.}
          
\cbb{aggregation: max agg size}{int}{-1}{Maximum size of an aggregate (-1 means unlimited).}
          
\cbb{aggregation: brick x size}{int}{2}{Number of points for x axis in "brick" aggregation (limited to 3).}
          
\cbb{aggregation: brick y size}{int}{2}{Number of points for y axis in "brick" aggregation (limited to 3).}
          
\cbb{aggregation: brick z size}{int}{2}{Number of points for z axis in "brick" aggregation (limited to 3).}
          
\cbb{aggregation: Dirichlet threshold}{double}{0.0}{Threshold for determining whether entries are zero during Dirichlet row detection.}
          
\cbb{aggregation: export visualization data}{bool}{false}{Export data for visualization post-processing.}
          
\cbb{aggregation: output filename}{string}{""}{Filename to write VTK visualization data to.}
          
\cbb{aggregation: output file: time step}{int}{0}{Time step ID for non-linear problems.}
          
\cbb{aggregation: output file: iter}{int}{0}{Iteration for non-linear problems.}
          
\cbb{aggregation: output file: agg style}{string}{Point Cloud}{Style of aggregate visualization.}
          
\cbb{aggregation: output file: fine graph edges}{bool}{false}{Whether to draw all fine node connections along with the aggregates.}
          
\cbb{aggregation: output file: coarse graph edges}{bool}{false}{Whether to draw all coarse node connections along with the aggregates.}
          
\cbb{aggregation: output file: build colormap}{bool}{false}{Whether to output a random colormap in a separate XML file.}
          

% ==================== REBALANCING ====================
\section{Rebalancing options}
\label{sec:options_rebalancing}


\cbb{repartition: enable}{bool}{false}{Rebalancing on/off switch.}
          
\cbb{repartition: partitioner}{string}{"zoltan2"}{Partitioning package to use. Possible values: "zoltan" (\zoltan{} library), "zoltan2" (\zoltantwo{} library).}
          
\cba{repartition: params}{\parameterlist}{Partitioner parameters. \muelu passes them directly to the appropriate package library.}
          
\cbb{repartition: start level}{int}{2}{Minimum level to run partitioner. \muelu does not rebalance levels finer than this one.}
          
\cbb{repartition: min rows per proc}{int}{800}{Minimum number of rows per processor. If the actual number if smaller, then rebalancing occurs.}
          
\cbb{repartition: max imbalance}{double}{1.2}{Maximum nonzero imbalance ratio. If the actual number is larger, the rebalancing occurs.}
          
\cbb{repartition: remap parts}{bool}{true}{Postprocessing for partitioning to reduce data migration.}
          
\cbb{repartition: rebalance P and R}{bool}{false}{Explicit rebalancing of R and P during the setup. This speeds up the solve, but slows down the setup phases.}
          

% ==================== MULTIGRID ====================
\section{Multigrid algorithm options}
\label{sec:options_mg}

\begin{table}[h!]
  \begin{center}
    \begin{tabular}{p{3.5cm} p{11cm}}
      \toprule
      \verb!sa!         & Classic smoothed aggregation~\cite{VMB1996} \\
      \verb!unsmoothed! & Aggregation-based, same as \verb!sa! but without damped Jacobi prolongator improvement step \\
      \verb!pg!         & Prolongator smoothing using $A$, restriction smoothing using $A^T$, local damping factors~\cite{ST2008} \\
      \verb!emin!       & Constrained minimization of energy in basis functions of grid transfer operator~\cite{WTWG2014,OST2011} \\
      \bottomrule
    \end{tabular}
    \caption{Available multigrid algorithms for generating grid transfer matrices. }
\label{t:mgs}
  \end{center}
\end{table}


\cbb{multigrid algorithm}{string}{"sa"}{Multigrid method. Possible values: see Table~\ref{t:mgs}.}
          
\cbb{sa: damping factor}{double}{1.33}{Damping factor for smoothed aggregation.}
          
\cbb{sa: use filtered matrix}{bool}{true}{Matrix to use for smoothing the tentative prolongator. The two options are: to use the original matrix, and to use the filtered matrix with filtering based on filtered graph used for aggregation.}
          
\cbb{filtered matrix: use lumping}{bool}{true}{Lump (add to diagonal) dropped entries during the construction of a filtered matrix. This allows user to preserve constant nullspace.}
          
\cbb{filtered matrix: reuse eigenvalue}{bool}{true}{Skip eigenvalue calculation during the construction of a filtered matrix by reusing eigenvalue estimate from the original matrix. This allows us to skip heavy computation, but may lead to poorer convergence.}
          
\cbb{emin: iterative method}{string}{"cg"}{Iterative method to use for energy minimization of initial prolongator in energy-minimization. Possible values: "cg" (conjugate gradient), "gmres" (generalized minimum residual), "sd" (steepest descent).}
          
\cbb{emin: num iterations}{int}{2}{Number of iterations to minimize initial prolongator energy in energy-minimization.}
          
\cbb{emin: num reuse iterations}{int}{1}{Number of iterations to minimize the reused prolongator energy in energy-minimization.}
          
\cbb{emin: pattern}{string}{"AkPtent"}{Sparsity pattern to use for energy minimization. Possible values: "AkPtent".}
          
\cbb{emin: pattern order}{int}{1}{Matrix order for the "AkPtent" pattern.}
          

% ==================== REUSE ====================
\section{Reuse options}
\label{sec:options_reuse}

Reuse options are a way for a user to speed up the setup stage of multigrid.
The main requirement to use reuse is that the matrix' graph structure does not
change. Only matrix values are allowed to change.

The reuse options control the degree to which multigrid hierarchy is preserved
for a subsequent setup call.

In addition, please note that not all combinations of multigrid algorithms and
reuse options are valid, or even make sense. For instance, the "emin" reuse
option should only be use with "emin" multigrid algorithm.

Table~\ref{t:reuse_types} contains the information about different reuse
options. The options are ordered in increasing number of reuse components, from
the no reuse to the full reuse ("full").

\begin{table}[h!]
  \begin{center}
    \begin{tabular}{p{3.0cm} p{12cm}}
      \toprule
      \verb!none!   & No reuse \\
      \verb!S!      & Reuse only the symbolic information of the level smoothers. \\
      \verb!tP!     & Reuse tentative prolongator. The graphs of smoothed
                      prolongator and matrices in Galerkin product are reused
                      only if filtering is not being used ({\it i.e.}, either
                      \verb!sa: use filtered matrix! or \verb!aggregation: drop tol! is false) \\
      \verb!emin!   & Reuse old prolongator as an initial guess to energy
                      minimization, and reuse the prolongator pattern \\
      \verb!RP!     & Reuse smoothed prolongator and restrictor. Smoothers are
                      recomputed.  \ding{42} \verb!RP! should reuse matrix graphs for
                      matrix-matrix product, but currently that is disabled as only \epetra{}
                      supports it. \\
      \verb!RAP!    & Recompute only the finest level smoothers, reuse all other operators \\
      \verb!full!   & Reuse everything \\
      \bottomrule
    \end{tabular}
    \caption{Available coarsening schemes. }
\label{t:reuse_types}
  \end{center}
\end{table}


\cbb{reuse: type}{string}{"none"}{Reuse options for consecutive hierarchy construction. This speeds up the setup phase, but may lead to poorer convergence. Possible values: see Table~\ref{t:reuse_types}.}
          

% ==================== MISCELLANEOUS ====================
\section{Miscellaneous options}


\cba{export data}{\parameterlist}{Exporting a subset of the hierarchy data in a
      file. Currently, the list can contain any of the following parameter
      names (``A'', ``P'', ``R'', ``Nullspace'', ``Coordinates'') of type \texttt{string}
      and value ``\{levels separated by commas\}''. A
      matrix/multivector with a name ``X'' is saved in two or three
      three MatrixMarket files: a) data is saved in
      \textit{X\_level.mm}; b) its row map is saved in
      \textit{rowmap\_X\_level.mm}; c) its column map (for matrices) is saved in
      \textit{colmap\_X\_level.mm}.}
          
\cbb{print initial parameters}{bool}{true}{Print parameters provided for a hierarchy construction.}
          
\cbb{print unused parameters}{bool}{true}{Print parameters unused during a hierarchy construction.}
          
\cbb{transpose: use implicit}{bool}{false}{Use implicit transpose for the restriction operator.}
          


    %-----------------------------%
    % \chapter{Performance}
    % In practice, it can be very challenging to find an appropriate set of multigrid
parameters for a specific problem, especially if few details are known about the
underlying linear system. In this Chapter, we provide some advice for improving
multigrid performance.

\begin{mycomment}
For optimizing multigrid parameters, it is highly recommended to set the
verbosity to \verb|high| or \verb|extreme| for \muelu{} to output more
information (e.g., for the effect of the chosen parameters to the aggregation
and coarsening process).
\end{mycomment}

Some general advice:
\begin{itemize}
  \item
    Choose appropriate iterative linear solver (e.g., GMRES for non-symmetric problems).

  \item
    Start with the recommended settings for particular problem types. See
    Table~\ref{t:problem_types}.

  \item
    Choose reasonable basic multigrid parameters
    (see~\S\ref{sec:options_general}), including maximum number of multigrid
    levels (\texttt{max levels}) and maximum allowed coarse size of the problem
    (\texttt{coarse: max size}). Take fine level problem size and sparsity
    pattern into account for a reasonable choice of these parameters.

  \item
    Select an appropriate transfer operator strategy
    (see~\S\ref{sec:options_mg}). For symmetric problems, you should start with smoothed
    aggregation multigrid. For non-symmetric problems, a Petrov-Galerkin smoothed
    aggregation method is a good starting point, though smoothed aggregation may
    also perform well.

  \item
    Enable implicit restrictor construction (\texttt{transpose:} \texttt{use implicit}) for symmetric
    problems.

  \item
    Find good level smoothers (see~\S\ref{sec:options_smoothing}). If a problem
    is symmetric positive definite, choose a smoother with a matrix-vector
    computational kernel, such as the Chebyshev polynomial smoother. If you are
    using relaxation smoothers, we recommend starting with optimizing the
    damping parameter. Once you have found a good damping parameter for your
    problem, you can increase the number of smoothing iterations.

  \item
    Adjust aggregation parameters if you experience bad coarsening ratios
    (see~\S\ref{sec:options_aggregation}). Particularly, try adjusting the
    minimum (\texttt{aggregation:} \texttt{min agg size}) and maximum
    (\texttt{aggregation:} \texttt{max agg size}) aggregation parameters. For a
    2D (3D) isotropic problem on a regular mesh, the aggregate size should be
    about 9 (27) nodes per aggregate.

  \item
    Replace a direct solver with an iterative method (\texttt{coarse: type}) if
    your coarse level solution becomes too expensive (see~\S\ref{sec:options_smoothing}).
\end{itemize}

Some advice for parallel runs include:
\begin{enumerate}
  \item
    Enable matrix rebalancing when running in parallel (\texttt{repartition:}
    \texttt{enable}).

  \item
    Use smoothers invariant to the number of processors, such as
    polynomial smoothing, if possible.

  \item
    Use \texttt{uncoupled} aggregation instead of \texttt{coupled}, as later
    requires significantly more communication.

  \item
    Adjust rebalancing parameters (see~\S\ref{sec:options_rebalancing}). Try
    choosing rebalancing parameters so that you end up with one processor on the
    coarsest level for the direct solver (this avoids additional communication).

  \item
    Enable implicit rebalancing of prolongators and restrictors
    (\texttt{repartition: rebalance P and R}).
\end{enumerate}


    %\nocite{*}

    % ---------------------------------------------------------------------- %
    % References
    %
    \clearpage
    % If hyperref is included, then \phantomsection is already defined.
    % If not, we need to define it.
    \providecommand*{\phantomsection}{}
    \phantomsection
    \addcontentsline{toc}{chapter}{References}
    \bibliographystyle{plain}
    \bibliography{ifpack2guide}


    % ---------------------------------------------------------------------- %
    %
    \appendix
    \input{appendix}
    %\chapter{Historical Perspective}
	%\input{CommonHistory}


    %\chapter{Some Other Appendix}
	%\input{CommonAppendix}

    % \printindex

    %
% This is an example of how to create the distribution page. Some
% distributions are required by Sandia; e.g. the housekeeping copies.
% Depending on the type of report; e.g. CRADA, Patent Caution, etc.
% additional distribution lines may have to be added. See the
% "Guide for Preparing SAND Reports"
%
% SANDdistribution takes CA or NM as an optional argument. If given,
% the approrpiate housekeeping copies are inserted autmatically.
% Inside the SANDdistribution environment, several commands can be used
% insert the distributions for CRADA, LDRD, etc. See example below.
%
% You can leave the CA or NM option off and not use any of the SANDdist*
% commands. This will allow you to create a distribution list manually.
%
\begin{SANDdistribution}[NM]
    % Housekeeping copies necessary for every unclassified report:
    % \SANDdistCRADA	% If this report is about CRADA work
    % \SANDdistPatent	% If this report has a Patent Caution or Patent Interest
    % \SANDdistLDRD	% If this report is about LDRD work

    % Some external Addresses

    \SANDdistExternal{1}{Roscoe A. Bartlett\\ Oak Ridge National Laboratory \\ P.O. Box 2008 \\ Oak Ridge, Tennessee 37831 \\ United States }
    % \SANDdistExternal{3}{Some Address\\ and street\\City, State}
    % \SANDdistExternal{12}{Another Address\\ On a street\\City, State\\U.S.A.}
    %\bigskip

    % The following MUST BE between the external and internal distributions!
    % \SANDdistClassified % If this report is classified

    % Internal Addresses
    \SANDdistInternal{1}{9018}{Central Technical Files}{8945-1}
    \SANDdistInternal{2}{0899}{Technical Library}{9610}
    \SANDdistInternal{2}{0612}{Review \& Approval Desk}{4916}
    \SANDdistInternal{1}{0897}{Todd S. Coffey}{1543}
    \SANDdistInternal{1}{1320}{Curtis C. Ober}{1442}
    \SANDdistInternal{1}{1318}{Roger P. Pawlowski}{1444}
    
    % Example of a mail channel use (instead of a mail stop)
    % \SANDdistInternalM{1}{M9999}{Someone}{01234}

\end{SANDdistribution}


\end{document}
